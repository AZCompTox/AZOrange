 \title{AZOrange Consensus Algorithm \\User Guide and Developer Documentation}
\author{
        Philip Werner \\
         Research and Development Information (RDI)\\
}
\date{\today}

\documentclass[12pt]{article}

\begin{document}
\maketitle

\newpage

\section{Introduction}

The consensus model uses multiple machine learning algorithms when making a prediction.
This method is useful in domains which are too complex for a single
algorithm to make an overall high prediction rate. A single algorithm might cover a part
of the domain very well, whilst being less precise in others, thus leading to less accurate
prediction rate. Instead of trying to overcome the problem of covering the whole domain
accurately using one algorithm, the consensus model relies on weighted decisions among
multiple ones. 

\section{Requirements}

\begin{list}{*}{}
\item Both classification and regression problems need to be supported.
\item Regression problems should support a generic mathematical expression.
\item Default behaviour for regression problems should an average value over all learners.
\item The regression expression should support weighting of the learners, expressed as a function.
\item Classification problems should support a generic logical expression
\item Default behaviour for classification problems using an uneven number of learners 
  should be majority vote.
\item Default behaviour for classification problems using an even number of learners
  should be average of N probabilities, where N is the number of learners.
\end{list}

\newpage
\section{User Guide}\label{user}

\subsection{Regression problems}

Regression problems can be initialized in two ways. Either the user has a custom
expression, or default mode is used. The default mode will be explained first, given
the simplicity of the input. Default mode in regression means using an average value
of the learners supplied. In mathematical terms:

\begin{equation}
  \frac{1}{n} \sum_{i=1}^n learner_i
\end{equation}
\\

An example using three learners will illustrate how to use the consensus model in 
default mode for regression. We first start by creating a list of learner objects,
where the minimum number of objects to use is two.

\begin{verbatim}
learners = [AZorngRF.RFLearner(), 
            AZorngCvSVM.CvSVMLearner(),
	            AZorngCvANN.CvANNLearner()]
\end{verbatim}

This list of learners will then be used as input to the consensus learner. To get
the classifier, we call the learner with the training data.

\begin{verbatim}
consensusLearner = 
    AZorngConsensus.ConsensusLearner(learners = learners)

classifier = consensusLearner(trainingData)
\end{verbatim}

The classifier is now ready to be used! 

\begin{verbatim}
prediction = classifier(example)
\end{verbatim}

\newpage
In some cases, the average of the learners will not be adequate as an expression.
In these cases, the user can supply a custom mathematical expression, as long
as it conforms to valid python code. The initialization of the consensus model will
then differ from that of default behaviour. Instead of using a list of learners,
as explained above, the user must supply two arguments. A dictionary (also known
as a map) of the learners, and a mathematical expression with variables, given as a string.
The variables will be mapped to the learners through the dictionary supplied as
input to the consensus model.

We will use the same example as above, but instead provide our own custom 
average expression. We have three learners, which we need to map to variables
using a dictionary.

\begin{verbatim}
learners = { 'a': AZorngRF.RFLearner(), 
             'b': AZorngCvSVM.CvSVMLearner(),
	             'c': AZorngCvANN.CvANNLearner() }
\end{verbatim}

The learners will now map to the variables a, b and c. As such we can use them in
the mathematical expression we provide to the consensus model. The mathematical 
expression is given below.

\begin{verbatim}
expression = '(a + b + c)/3'
\end{verbatim}

We now initialize our consensus model, using these arguments, and create a
classifier by supplying training data. The classifier is then used as before.

\begin{verbatim}
consensusLearner = 
    AZorngConsensus.ConsensusLearner(learners = learners, 
                                     expression = expression)

classifier = consensusLearner(trainingData)

prediction = classifier(example)
\end{verbatim}

The consensus model also has the ability to use individual weights for each learner.
This can be used to determine the importance each learner should have when making
the consensus prediction. The weights are stored in a dictionary, mapping the 
learner variables to the weight function. 

\begin{verbatim}
learners = { 'a': AZorngRF.RFLearner(), 
             'b': AZorngCvSVM.CvSVMLearner(),
	             'c': AZorngCvANN.CvANNLearner() }

def weightfunction1(example):
   return example[0]*2

def weightfunction2(example):
   return example[0]*4

def weightfunction3(example):
   return example[0]*6

weights = { 'a': weightfunction1, 
             'b': weightfunction2,
	             'c': weightfunction3 }

expression = '(a + b + c)/3'

consensusLearner = 
    AZorngConsensus.ConsensusLearner(learners = learners, 
                                     expression = expression,
				                                     weights = weights)

classifier = consensusLearner(trainingData)

prediction = classifier(example)
\end{verbatim}

The example above will use the mathematical expression shown below when making a prediction.

\begin{equation}
  \sum_{i=1}^n learner(ex)_i*weightfunction_i(ex)
\end{equation}
\\


\newpage
\subsection{Classification problems}

There are few differences with regards to initialization compared to regression
models, as such, only these will be covered in this section. As with the regression
model, an example will help highlight the use.

If only a list of learners is specified as input, the consensus model will use
either a majority vote or return a result based on probability. The choice depends
on the number of learners provided; an even number will predict depending on probabilities,
while an uneven will use majority voting. Below is the code to initialize the consensus
learner using default behaviour. It is the training data, with its domain, which determines
if the problem is classification or regression. Care must be taken such that the learners
can return probabilities, if an even number of learners is provided to the consensus model.

\begin{verbatim}
learners = [AZorngRF.RFLearner(), 
            AZorngCvSVM.CvSVMLearner(),
	            AZorngCvANN.CvANNLearner()]
consensusLearner = 
    AZorngConsensus.ConsensusLearner(learners = learners)

classifier = consensusLearner(trainingData)
\end{verbatim}

\newpage
Classification problems can also use a custom logical expression to determine the outcome
of a prediction. The idea behind this logical expression is to mimic, as much as possible,
the use of if/else-if/else statements, found in most languages. The if/else-if/else
statements used with the consensus learner, is simply a list of strings, representing
logical expressions and their return statements, if the statement is true. We will now
convert a trivial if/else-if/else statement into the expected format of the consensus 
learner. Below is the trivial example:

\begin{verbatim}
if a == POS then
    return POS
else
    return NEG
end if
\end{verbatim}

And this is transformed into the following list of strings.

\begin{verbatim}
[``a == POS -> POS'', 
 ``-> NEG'']
\end{verbatim}

A more complex example:

\begin{verbatim}
if a == POS and b == POS then
    return POS
else if a == NEG and b == NEG then
    return NEG
else
    return UNKNOWN
end if
\end{verbatim}

And this is transformed into the following list of strings.

\begin{verbatim}
[``a == POS and b == POS -> POS'', 
 ``a == NEG and b == NEG -> NEG'', 
 ``-> UNKNOWN``]
\end{verbatim}

\newpage
Given the last expression, a dictionary mapping the variables to learners need to
be specified. The complete code example is shown below.

\begin{verbatim}

learners = { 'a': AZorngRF.RFLearner(), 
             'b': AZorngCvSVM.CvSVMLearner(),
	             'c': AZorngCvANN.CvANNLearner() }

expression = 
  [``a == POS and b == POS -> POS'', 
   ``a == NEG and b == NEG -> NEG'', 
   ``-> UNKNOWN``]

consensusLearner = 
    AZorngConsensus.ConsensusLearner(learners = learners, expression = expression)

classifier = consensusLearner(trainingData)

prediction = classifier(example)
\end{verbatim}

\newpage
\section{Developer Documentation}\label{developer}

\end{document}

